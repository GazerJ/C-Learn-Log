\documentclass{article}
\usepackage{CJKutf8}
\usepackage{graphicx}
\usepackage{svg}
\raggedright
\begin{document}
\begin{CJK}{UTF8}{gkai}
\title{Makefile 学习笔记 }
\author{姜高晓}
\date{\today}
\maketitle
\begin{abstract}
	这是摘要
\end{abstract}

\section{背景}
\subsection{什么是Makefile}
 	MakeFile 可以认为是工程文件编译链接的指导书,指挥着编译工程文件的时候的顺序和规则。这可以使得我们的工程目录编译链接自动化,而且有条理。
\subsection{基本结构}
Makefile的基本结构如下
\\
目标:依赖文件名
\\
\ \ \ \ 命令
\begin{itemize}
	\item 目标:可以是输出文件,也可是标签
	\item 依赖文件:编译前要搞定的
	\item 命令:编译链接命令,也可使shell命令
\end{itemize}

\subsection{Makefile的执行流程}
第一规则:在Makefile执行的时候,默认第一个目标就是最终目标main。
\\
\\
我们执行 make 命令时,只有修改过的源文件或者是不存在的目标文件会进行重建,而那些没有改变的文件不用重新编译,这样在很大程度上节省时间,提高编程效率。小的工程项目可能体会不到,项目工程文件越大,效果才越明显。
\\
\\
ps: \ \  .o文件不要着急清理,等需要清理的时候在清理,这样就可以保证编译时的速度
\\
\\
伪目标:clean, 在执行make clean的时候清除一些垃圾,他和main是并列关系不需要依赖文件。
\section{主要内容}
\subsection{参数引用和通配符}
每一个目标-命令结构都可以当作一个脚本,这里可以用\$ ,引用参数。
\begin{itemize}
	\item \$\# 是传给脚本的参数个数
	\item \$0  是脚本本身的名字
	\item \$1  是传递给该shell脚本的第一个参数
	\item \$2  是传递给该shell脚本的第二个参数
	\item \$@ 是传给脚本的所有参数的列表,代表目标文件(target) 
	\item \$* 是以一个单字符串显示所有向脚本传递的参数,与位置变量不同,参数可超过 9 个
	\item \$\$ 是脚本运行的当前进程ID号
	\item \$? 是显示最后命令的退出状态, 0 表示没有错误,其他表示有错误
	\item @这个符串通常用在“规则”行中,表示不显示命令本身,而只显示它的结果
	\item \$\^{}代表所有的依赖文件(components)
	\item \$\textless 代表第一个依赖文件(components中最左边的那个)。
\end{itemize}
\$@ 是目标  用以  -o \$@ 
	\\ 
\$\^{}是依赖    需要处理的文件 

Makefile支持shell中的通配符,*,?,[ ].除此之外还支持\%.
\begin{itemize}
	\item * :任意个任意字符。
	\item ? : 一个任意字符。
	\item \  [ ] : 把指定的字符放入方括号中
	\item \% : 匹配字符,例如依赖中依赖  x1.o x2.o x3.o  可以简单写一个  \%.o 的目标,统一处理这些依赖
\end{itemize}
\subsection{变量赋值}
\begin{itemize}
	\item 简单赋值 ( := ) 编程语言中常规理解的赋值方式,只对当前语句的变量有效。
	\item 递归赋值 ( = ) 赋值语句可能影响多个变量,所有目标变量相关的其他变量都受影响。
	\item 条件赋值 ( ?= ) 如果变量未定义,则使用符号中的值定义变量。如果该变量已经赋值,则该赋值语句无效。
	\item 追加赋值 ( += ) 原变量用空格隔开的方式追加一个新值。
\end{itemize}

\subsection{GCC}


\section{总结}
\subsection{Subsection title}
        这里是总结



\begin{thebibliography}{12345}%引文
\bibitem{vicsek} http://c.biancheng.net/view/7097.html

\bibitem{vicsek} https://blog.csdn.net/u014285914/article/details/124048754 

\end{thebibliography}
\end{CJK}
\end{document}




