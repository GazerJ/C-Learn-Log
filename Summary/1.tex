\documentclass{article}
%\documentclass[aps,prl,reprint,groupedaddress]{revtex4-1}
\usepackage{CJKutf8}
%\usepackage[utf8]{inputenc}
\usepackage{graphicx}
\usepackage{svg}
\raggedright

\begin{document}

\begin{CJK}{UTF8}{gkai}

\title{C++ 复习 }
\author{ GazerJ}
%\altaffiliation{Department of Physics, Xiamen University, Xiamen 361005, People’s Republic of China}
%\email{ gazerj@foxmail.com}
%\affiliation{Department of Physics, Xiamen University, Xiamen 361005, People’s Republic of China}
%\author{...} % 一个作者有多个单位
\date{\today}
\maketitle
\begin{abstract}
	本文是本人复习C++,所写,旨在学习和总结
\end{abstract}

\section{基础知识}
	\subsection{存储区域}
		计算机有5个存储区域:
	\begin{itemize}
	\item 堆,程序员自己申请的内存储存的区域
	\item 栈,编译器管理的区域,不用时自动销毁
	\item 全局静态区,全局变量,static关键字修饰的变量
	\item 常量区,const修饰的变量 存储的区域,不可修改
	\item 代码区 函数的二进制储存区
	\end{itemize}
\section{面向对象}

\section{多线程}


\begin{thebibliography}{12345}%引文
\bibitem{vicsek} Vicsek T


\end{thebibliography}
\end{CJK}
\end{document}




